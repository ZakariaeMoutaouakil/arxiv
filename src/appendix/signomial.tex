\section{Signomial Programming}\label{sec:signomial-programming}

\subsection{Definitions}\label{subsec:definitions}
In signomial programming, the building block is a \textbf{monomial} function which is a function of the form
\[f(\mathbf{x}) = c x_1^{a_1} x_2^{a_2} \cdots x_m^{a_m}\]

where:
\begin{itemize}
    \item $c > 0$ is a positive coefficient,
    \item $\mathbf{x} = (x_1, x_2, \ldots, x_m)$ are positive variables,
    \item $a_1, a_2, \ldots, a_m$ are real exponents (not necessarily non-negative).
\end{itemize}
Building on top of this definition, we can define two types of functions.
A \textbf{posynomial} function is a sum of monomials, while a \textbf{signomial} function is a linear combination of monomials (meaning that the multiplicative coefficients can be negative).

A signomial program (SP) is an optimization problem of the form:
\[
    \begin{aligned}
        & \text{minimize}   & & f_0(\mathbf{x}) \\
        & \text{subject to} & & f_i(\mathbf{x}) \geq 0, \quad i = 1, \ldots, p \\
        &                   & & \mathbf{x} > 0
    \end{aligned}
\]
where:

\begin{itemize}
    \item $\mathbf{x} = (x_1, \ldots, x_m)$ is the vector of optimization variables,
    \item $f_0, f_1, \ldots, f_m$ are signomial functions.
\end{itemize}

Signomials programs generalize the well-known geometric programs that are much easier to solve, since they can be reduced to convex optimization problems.
However, signomial programs are generally non-convex optimization problems and can be challenging to solve globally.
Various techniques, such as successive convex approximation or branch-and-bound methods, are often employed to find solutions to SPs.

