\section{Certified Radius Estimation in the Discrete Case}\label{sec:discrete}

\subsection{Definitions and Notations}
Following the same footsteps in , we have a vector of counts $X = (X_1, \ldots, X_m)$ that follows a multinomial distribution with parameters $n$ and $p = (p_1, \ldots, p_m)$, where $n$ is the number of samples (which is fixed for our purposes) and $p_k = \mathbb{P}(f(x + \epsilon) = k)$ are the sole unknown parameters.
In general, we wish to estimate a lower confidence bound (the development for an upper confidence bound is identical but uninteresting for our intents given the reasons we layed out in the previous section) for a real-valued function $\theta\coloneqq g(p)$ of the multinomial parameter $p$.
We assume that the multinomial parameter $p$ lives in a constraint space denoted as $\chi^{m-1}$ which is a subset of $\Delta^{m-1}$.
We also denote by $\chi^{m-1}_n\coloneqq\chi^{m-1}\cap\Delta^{m-1}_n$ the discrete constraint space, i.e. the values that can taken empirically by $p$ given $n$ trials.

The maximum likelihood estimator (MLE) of $\theta$ is given by $\Hat{\theta}=g(\Hat{p})$ where $\Hat{p} = \frac{X}{n}=\left(\frac{X_1}{n}, \cdots, \frac{X_m}{n}\right)$ is the MLE of the multinomial parameter $p$.
Define $\Theta\coloneqq g(\chi^{m-1})$ (respectively $\Hat{\Theta}\coloneqq g(\chi^{m-1}_n)$) as the set of all possible values that can be taken by $\theta$ (respectively by $\Hat{\theta}$).
Denote by $\Tilde{\theta}\in\Hat{\Theta}$ the observed value of $\theta$ in our data, and by $\Pi(\cdot|p)$ the cumulative distribution function (CDF) of $\Hat{\theta}$ given a multinomial parameter $p\in\Delta^{m-1}$.
For any $L\in\Theta$, define
\begin{equation}
    \Pi(L)\coloneqq1-\inf_{\substack{p\in\Delta^{m-1}\\g(p)\leq L}}\Pi(\Tilde{\theta}|p).\label{eq:Pi}
\end{equation}

Denote by $\Omega^{m-1}_n\coloneqq\{x\in\mathbb{R}^m : x_i\geq0,\sum_{i=1}^m x_i = n\}$ the sample space of the multinomial distribution.
For any $p\in\Delta^{m-1}$ and $x\in\Omega^{m-1}_n$, we write
\[
    \binom{n}{x}\coloneqq\binom{n}{x_1}\cdots\binom{n}{x_m},\quad p^x\coloneqq p_1^{x_1}\cdots p_m^{x_m}.
\]
Hence, we can explicitly write the CDF of $\Hat{\theta}$ as
\[
    \Pi(\Tilde{\theta}|p) = \sum_{\substack{x\in\Omega^{m-1}_n\\g\left(\frac{x}{n}\right)\leq\Tilde{\theta}}}\binom{n}{x}p^x.
\]

For any $p\in\Delta^{m-1}$ such that $m\geq3$ (which we assume), we define the reduced multinomial parameter $q\in\Delta^2$ (which depends implicilty on $p$) by $q_1=p_1$, $q_2=p_2$ and $q_3=1-p_1-p_2$.
It can be proven by simple algebraic manipulations that if the function $g$ depends only on the parameters $p_1$ and $p_2$, then the CDF of $\Hat{\theta}$ can be simplified to
\[
    \Pi(\Tilde{\theta}|p)=\Pi(\Tilde{\theta}|q)= \sum_{\substack{x\in\Omega^{2}_n\\g\left(\frac{x}{n}\right)\leq\Tilde{\theta}}}\binom{n}{x}q^x.
\]
It is also worth noting that $\Pi(\Tilde{\theta}|q)$ is a posynomial function in the parameter $q$, a fact that we will use later.
Finally, the quantity $\Pi(L)$ can be simplified to
\[
    \Pi(L) = 1 - \inf_{\substack{q\in\Delta^{2}\\g(q)\leq L}}\Pi(\Tilde{\theta}|q).
\]

\subsection{First Radius Estimation}\label{subsec:first-radius-estimation}
In the case of the first radius, our constraint space $\chi^{m-1}$ is the entire simplex $\Delta^{m-1}$.
Therefore, the discrete constraint set and the discrete simplex set coincide.
The sets $\Theta$ and $\Hat{\Theta}$ are equal to $[ -1, 1 ]$ and $\{-\frac{n}{n},\frac{n-1}{n}, \dots,-\frac{1}{n},0,\frac{1}{n},\frac{2}{n},\dots,\frac{n}{n}\}$ respectively.
The observed value $\Tilde{\theta}\in\Hat{\Theta}$ can be written as $\Tilde{\theta} = \frac{k}{n}$ for some $k\in\llbracket -n, n \rrbracket$.
Without loss of generality, we can assume that the two classes with the highest probabilities are respectively classes $1$ and $2$ (i.e. $y=1$ and $j=2$).
Hence, the CDF of $\Hat{\theta}$ can be rewritten as
\[
    \Pi(\Tilde{\theta}|p) =\Pi(\Tilde{\theta}|q)= \mathbb{P}(X_1 - X_2\leq k|q)=\sum_{x_2=0}^n\sum_{x_1=0}^{\min(k+x_2,n)}\binom{n}{x}q^x.
\]
For a given confidence level $1-\alpha$, the lower confidence bound on $\theta$ is given by
\begin{equation}
    \underline{\Hat{\theta}} = \inf\left\{ L\in\Theta : \Pi(L) = \alpha \right\}.\label{eq:lower-confidence-bound}
\end{equation}
Therefore, to compute the lower confidence bound $\underline{\Hat{\theta}}$, we need to solve two optimization problems in the equations~\eqref{eq:Pi} and~\eqref{eq:lower-confidence-bound}.
Starting by the easy one, suppose we can compute $\Pi(L)$ for any $L\in\Theta$, it is clear that the function $\Pi$ is nondecreasing in $L$, which means that we can find $\underline{\Hat{\theta}}$ by binary search.

Computing $\Pi(L)$ demands that we solve the following optimization problem:
\begin{equation}
    \inf_{\substack{q\in\Delta^{2}\\q_1-q_2\leq L}}\Pi(\Tilde{\theta}|q).\label{eq:Pi-optimization}
\end{equation}

This optimization problem~\eqref{eq:Pi-optimization} falls under the category of \textbf{signomial optimization problems}, a kind of optimization problems that are extensively studied in the literature.
Details on how to solve the signomial optimization problem can be found in Appendix~\ref{sec:signomial-optimization}.
The high-level algorithm for finding $\underline{\Hat{\theta}}$ is given in Algorithm~\ref{alg:first-margin-estimation}.
\begin{algorithm}[h]
    \DontPrintSemicolon % Removes semicolons at the end of lines
    \KwIn{A vector of counts $X$, a tiny threshold $\epsilon>0$ and a routine \texttt{SolveSignomial(L)} that solves the signomial optimization problem~\eqref{eq:Pi-optimization} for any $L\in[-1,1]$}
    \KwOut{The lower confidence bound $\underline{\Hat{\theta}}$ at confidence level $1-\alpha$}
    Compute the observed value $\Tilde{\theta}\shortleftarrow\frac{X_1-X_2}{n}$\;
    Set initial bounds: $\text{left}\shortleftarrow0$, $\text{right}\shortleftarrow\Tilde{\theta}$ \;
    \While{$\text{right}-\text{left}<\epsilon$}{
        $L\shortleftarrow\frac{\text{right}+\text{left}}{2}$\;
        $\text{Solution}\shortleftarrow$ \texttt{SolveSignomial(L)}\;
        \If{$\text{Solution}<1-\alpha$}{
            $\text{right}\shortleftarrow L$\;
        }
        \Else{
            $\text{left}\shortleftarrow L$\;
        }
    }
    \KwRet{$\text{left}$}
    \caption{First Margin Estimation in the Discrete Case}\label{alg:first-margin-estimation}
\end{algorithm}

Solving signomial problems can be time-consuming, this is the reason why the next lemma can helpful in justifying suboptimal but faster methods in the first iterations of the binary search.
\begin{lemma}
    \label{lemma:suboptimal}
    If there exists $q\in\Delta^{2}$ such that $q_1-q_2\leq L$ and $\Pi(\Tilde{\theta}|q)\leq 1-\alpha$, then $\underline{\Hat{\theta}} \leq L$.
\end{lemma}

It is worth noting that this lemma holds true for both first and second radii.
The lemma above has the following implication: if we suboptimally solve the optimization problem~\eqref{eq:Pi-optimization} for a given $L\in\Theta$, and if we find a value that is less than $1-\alpha$, then the lower confidence bound we are looking for resides on the right side of $L$, which narrows down the search space to the left subspace of $L$.
In the first iterations of the binary search, our candidate $L$ is likely to be far away from the optimal solution $\underline{\Hat{\theta}}$, which means that solutions of the optimization problem~\eqref{eq:Pi-optimization} are likely to correspond to values that are less $1-\alpha$, therefore we can get away with suboptimally solving the problem.
In practice, bayesian optimization algorithms often work well and are reasonably fast for the problem~\eqref{eq:Pi-optimization}, and we implement an early stopping rule to stop optimizing further if we fall below $1-\alpha$.
The final algorithm is summarized in Algorithm~\ref{alg:first-margin-estimation-fast}.
\begin{algorithm}[h]
    \DontPrintSemicolon % Removes semicolons at the end of lines
    \KwIn{A vector of counts $X$, a tiny threshold $\epsilon>0$, a routine \texttt{SolveSignomial(L)} that solves the signomial optimization problem~\eqref{eq:Pi-optimization} for any $L\in[-1,1]$ and another routine \texttt{FastSolveSignomial(L)} that does the same thing but in a faster and possibly suboptimal way}
    \KwOut{The lower confidence bound $\underline{\Hat{\theta}}$ at confidence level $1-\alpha$}
    Compute the observed value $\Tilde{\theta}\shortleftarrow\frac{X_1-X_2}{n}$\;
    Set initial bounds: $\text{left}\shortleftarrow0$, $\text{right}\shortleftarrow\Tilde{\theta}$, $\text{close}\shortleftarrow\text{False}$ \;
    \While{$\text{right}-\text{left}<\epsilon$}{
        $L\shortleftarrow\frac{\text{right}+\text{left}}{2}$\;
        \If{$\text{close}$}{
            $\text{Solution}\shortleftarrow$ \texttt{SolveSignomial(L)}\;
        }
        \Else{
            $\text{Solution}\shortleftarrow$ \texttt{FastSolveSignomial(L)}\;
        }
        \If{$\text{Solution}<1-\alpha$}{
            $\text{right}\shortleftarrow L$\;
        }
        \Else{
            $\text{close}\shortleftarrow\text{True}$\;
            $\text{left}\shortleftarrow L$\;
        }
    }
    \KwRet{$\text{left}$}
    \caption{First Margin Estimation in the Discrete Case (Fast Version)}\label{alg:first-margin-estimation-fast}
\end{algorithm}

\subsection{Second Radius Estimation}\label{subsec:second-radius-estimation}
In the same way as the last section, we need to solve the following optimization problem
\begin{equation}
    \inf_{\substack{q\in\Delta^{2}\\\Phi^{-1}(q_1)-\Phi^{-1}(q_2)\leq L}}\Pi(\Tilde{\theta}|q).\label{eq:non-signomial-optimization}
\end{equation}

Since $\Phi^{-1}$ is not a posynomial nor a signomial function, the optimization problem~\eqref{eq:non-signomial-optimization} cannot be classified as a signomial optimization problem, due to the non-signomial constraint $\Phi^{-1}(q_1)-\Phi^{-1}(q_2)\leq L$.
To circumvent this problem, the following lemma shows that we can substitue the function $\Phi^{-1}$ with its taylor series approximation which is a polynomial (and in particular, a posynomial) function.
\begin{lemma}
    \label{lemma:approximation}
    Let $\Phi^{-1}_M$ be the taylor series approximation of $\Phi^{-1}$ up to order $M$.
    If we assume that $p_1\geq1/2$, then
    \[
        M(\Phi^{-1}\circ\hat{F},x) \geq M(\Phi^{-1}_M\circ\hat{F},x)
    \]
\end{lemma}
This lemma guarantees that replacing $\Phi^{-1}$ with its taylor series approximation preserves the conservativeness of the certified radius.
The new margin has the advantage that the following optimization problem is a signomial optimization problem
\begin{equation}
    \inf_{\substack{q\in\Delta^{2}\\\Phi^{-1}_M(q_1)-\Phi^{-1}_M(q_2)\leq L}}\Pi(\Tilde{\theta}|q),\quad\text{ with }\quad\Pi(\Tilde{\theta}|q)= \sum_{\substack{x\in\Omega^{2}_n\\\Phi^{-1}_M\left(\frac{x_1}{n}\right)-\Phi^{-1}_M\left(\frac{x_2}{n}\right)\leq\Tilde{\theta}}}\binom{n}{x}q^x.\label{eq:approximated-signomial-optimization}
\end{equation}

In the same manner as before, in the first iterations of binary search we can suboptimally solve this optimization problem using techniques such as bayesian optimization.

A disadvantage of the taylor series approximation is that the lemma assumes that $p_1\geq1/2$ to maintain the conservativeness of the certified radius.
To get around this potentially untrue hypothesis, we can test it using the one-sided Clopper-Pearson interval.
If the test succeeds, we proceed with the taylor series approximation.
If not, as a fallback, we use the standard Clopper-Pearson interval with Bonferroni correction to estimate the certified radius.
This approach is summarized in Algorithm~\ref{alg:second-margin-estimation}.
\begin{algorithm}[h]
    \DontPrintSemicolon % Removes semicolons at the end of lines
    \KwIn{A vector of counts $X$, a tiny threshold $\epsilon>0$, a routine \texttt{SolveSignomial(L)} that solves the signomial optimization problem~\eqref{eq:approximated-signomial-optimization} for any $L\in[-1,1]$ and another routine \texttt{FastSolveSignomial(L)} that does the same thing but in a faster and possibly suboptimal way}
    \KwOut{The lower confidence bound $\underline{\Hat{\theta}}$ at confidence level $1-\alpha$}
    Computer the lower confidence bound $\underline{p_1}$ using the one-sided Clopper-Pearson interval at level $\frac{\alpha}{2}$\;
    \If{$p_1>\frac{1}{2}$}{
        Compute the observed value $\Tilde{\theta}\shortleftarrow\Phi^{-1}\left(\frac{X_1}{n}\right)-\Phi^{-1}\left(\frac{X_2}{n}\right)$\;
        Continue alongside Algorithm~\ref{alg:first-margin-estimation-fast} while replacing $\alpha$ with $\frac{\alpha}{2}$\;
    }
    \Else{
        Compute the lower confidence bound using the standard Bonferonni algorithm (Algorithm~\ref{alg:bonferroni})\;
    }
%    \KwRet{$\text{left}$}
    \caption{Second Margin Estimation in the Discrete Case}\label{alg:second-margin-estimation}
\end{algorithm}
